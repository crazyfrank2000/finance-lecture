\chapter{Duration}

%——————————————————————————————————%

\section{Key Duration Relationship}

\textbf{Duration leads the key relationship between bond price and time}
\textbf{债券久期关系 (Bond Duration Relationship)}

\subsection{基本久期}

久期是衡量债券价格对收益率变化敏感性的重要指标。其基本关系式为:

\begin{equation}
\frac{\Delta B}{B} = -D\Delta y
\end{equation}

\noindent 其中:
\begin{itemize}
    \item $\Delta B$ = 债券价格变化
    \item $B$ = 债券当前价格
    \item $D$ = 久期
    \item $\Delta y$ = 收益率变化
\end{itemize}

\textbf{关键要点:}
\begin{itemize}
    \item 久期衡量了债券价格对收益率变化的敏感程度
    \item 负号表示价格与收益率反向变动
    \item 久期越大,价格对收益率变化越敏感
\end{itemize}

\subsection{修正久期 (Modified Duration)}

当收益率按复利计算时,需要使用修正久期:

\begin{equation}
\Delta B = -\frac{BD\Delta y}{1 + y/m}
\end{equation}

修正久期定义为:
\begin{equation}
\text{修正久期} = \frac{D}{1 + y/m}
\end{equation}

\noindent 其中:
\begin{itemize}
    \item $m$ = 每年复利次数
    \item $y$ = 年收益率
    \item $D$ = 麦考利久期 (Macaulay Duration)
\end{itemize}




\textbf{麦考利久期的定义}

麦考利久期是债券现金流的加权平均到期时间:

\begin{equation}
D_{Mac} = \frac{\sum_{t=1}^{n} \frac{t \cdot CF_t}{(1+y/m)^{tm}}}{\sum_{t=1}^{n} \frac{CF_t}{(1+y/m)^{tm}}} = \frac{\sum_{t=1}^{n} t \cdot PV(CF_t)}{B}
\end{equation}

\noindent 其中:
\begin{itemize}
    \item $CF_t$ = 第t期现金流
    \item $PV(CF_t)$ = 第t期现金流现值
    \item $B$ = 债券当前价格
\end{itemize}

\textbf{修正久期的推导}

债券价格对收益率的一阶导数:

\begin{equation}
\frac{dB}{dy} = -\frac{1}{m} \sum_{t=1}^{n} \frac{t \cdot CF_t}{(1+y/m)^{tm+1}}
\end{equation}

整理得到:

\begin{equation}
\frac{dB}{dy} = -\frac{B \cdot D_{Mac}}{1 + y/m}
\end{equation}

因此修正久期定义为:

\begin{equation}
D_{Mod} = \frac{D_{Mac}}{1 + y/m}
\end{equation}

价格敏感性关系:

\begin{equation}
\frac{\Delta B}{B} = -D_{Mod} \cdot \Delta y
\end{equation}

\subsection{两种久期的关系与区别}

\subsubsection{数学关系}

\begin{equation}
D_{Mod} = \frac{D_{Mac}}{1 + y/m}
\end{equation}

\noindent 其中:
\begin{itemize}
    \item $D_{Mac}$ = 麦考利久期
    \item $D_{Mod}$ = 修正久期
    \item $y$ = 年收益率
    \item $m$ = 每年复利次数
\end{itemize}

\subsubsection{关键区别}

\begin{table}[h]
\centering
\begin{tabular}{|l|l|l|}
\hline
\textbf{特征} & \textbf{麦考利久期} & \textbf{修正久期} \\
\hline
定义 & 现金流加权平均时间 & 价格敏感性指标 \\
\hline
单位 & 年 & 百分比变化/收益率变化 \\
\hline
用途 & 理论分析 & 风险管理 \\
\hline
数值关系 & $D_{Mac} > D_{Mod}$ & $D_{Mod} < D_{Mac}$ \\
\hline
\end{tabular}
\end{table}

\subsubsection{实际影响因素}

调整因子 $\frac{1}{1 + y/m}$ 的影响:

\begin{itemize}
    \item \textbf{收益率水平}:收益率越高,修正久期相对麦考利久期的折扣越大
    \item \textbf{复利频率}:复利次数越高,调整幅度越小
    \item \textbf{数值示例}:
    \begin{itemize}
        \item 年收益率6\%,年复利1次:调整因子 = $\frac{1}{1.06} = 0.943$
        \item 年收益率6\%,半年复利2次:调整因子 = $\frac{1}{1.03} = 0.971$
        \item 年收益率6\%,连续复利:调整因子 = $e^{-0.06} = 0.942$
    \end{itemize}
\end{itemize}

\subsection{实际应用}

\subsubsection{风险管理应用}
\begin{itemize}
    \item \textbf{投资组合风险评估}:使用修正久期计算组合的利率风险暴露
    \item \textbf{对冲策略设计}:根据修正久期匹配资产负债的利率敏感性
    \item \textbf{VaR计算}:修正久期是利率VaR模型的核心参数
\end{itemize}

\subsubsection{投资决策应用}
\begin{itemize}
    \item \textbf{利率预期管理}:
    \begin{itemize}
        \item 预期利率上升 → 选择低修正久期债券
        \item 预期利率下降 → 选择高修正久期债券
    \end{itemize}
    \item \textbf{免疫策略}:匹配资产和负债的修正久期
\end{itemize}

\subsubsection{数值示例}

假设一个债券的麦考利久期为5年,年收益率为4\%,半年付息:

\begin{align}
D_{Mod} &= \frac{D_{Mac}}{1 + y/m} \\
&= \frac{5}{1 + 0.04/2} \\
&= \frac{5}{1.02} \\
&= 4.902 \text{年}
\end{align}

如果收益率上升100个基点(1\%),债券价格预期变化:
\begin{equation}
\frac{\Delta B}{B} = -4.902 \times 0.01 = -4.902\%
\end{equation}


\section{债券凸性 (Bond Convexity)}

\subsection{凸性的定义}

凸性是衡量债券价格对收益率变化的二阶敏感性指标:

\begin{equation}
C = \frac{1}{B} \frac{\partial^2 B}{\partial y^2} = \frac{\sum_{t=1}^{n} c_t t^2 e^{-yt}}{B}
\end{equation}

\subsection{改进的价格敏感性公式}

包含凸性的更准确关系式:

\begin{equation}
\frac{\Delta B}{B} = -D\Delta y + \frac{1}{2}C(\Delta y)^2
\end{equation}

\noindent 其中:
\begin{itemize}
    \item 第一项:久期效应(线性项)
    \item 第二项:凸性效应(二次项,总是正数)
\end{itemize}

\subsection{凸性的重要性}

\textbf{久期的局限性:}
久期假设价格与收益率呈线性关系,但实际上债券价格曲线是凸向原点的非线性关系。

\textbf{凸性的优势:}
\begin{itemize}
    \item 提供下行风险保护
    \item 增强上行收益潜力
    \item 收益率下降时的价格上升 > 收益率上升时的价格下降
\end{itemize}

\subsection{影响因素}

凸性大小受以下因素影响:
\begin{itemize}
    \item 到期时间越长 → 凸性越大
    \item 票息率越低 → 凸性越大
    \item 收益率水平越低 → 凸性越大
\end{itemize}

\section{20年期国债价格变化计算推导}

\subsection{债券基本参数}

给定条件:
\begin{itemize}
    \item 到期时间:$T = 20$ 年
    \item 收益率:$y = 4.5\% = 0.045$
    \item 票息率:$c = 4.5\% = 0.045$(假设)
    \item 面值:$F = \$100$
    \item 付息频率:$m = 2$(半年付息)
    \item 收益率变化:$\Delta y = -0.0001$(下降1BP)
\end{itemize}

\subsection{债券定价公式}

债券价格的一般公式:
\begin{equation}
B = \sum_{t=1}^{n} \frac{C}{(1+y/m)^t} + \frac{F}{(1+y/m)^n}
\end{equation}

其中:
\begin{itemize}
    \item $C = \frac{c \cdot F}{m} = \frac{0.045 \times 100}{2} = \$2.25$(每期付息)
    \item $n = T \times m = 20 \times 2 = 40$(总付息期数)
    \item $y/m = 0.045/2 = 0.0225$(每期收益率)
\end{itemize}

\subsection{麦考利久期计算}

麦考利久期定义:
\begin{equation}
D_{Mac} = \frac{1}{B} \sum_{t=1}^{n} \frac{t \cdot C_t}{(1+y/m)^t}
\end{equation}

其中 $C_t$ 是第$t$期的现金流。

计算过程:
\begin{align}
D_{Mac} &= \frac{1}{B} \left[ \sum_{t=1}^{39} \frac{t \cdot 2.25}{(1.0225)^t} + \frac{40 \cdot 102.25}{(1.0225)^{40}} \right]
\end{align}

\subsection{修正久期计算}

修正久期公式:
\begin{equation}
D_{Mod} = \frac{D_{Mac}}{1 + y/m} = \frac{D_{Mac}}{1 + 0.0225} = \frac{D_{Mac}}{1.0225}
\end{equation}

\subsection{凸性计算}

凸性定义:
\begin{equation}
C = \frac{1}{B} \sum_{t=1}^{n} \frac{t(t+1) \cdot C_t}{m^2 \cdot (1+y/m)^{t+2}}
\end{equation}

简化为:
\begin{equation}
C = \frac{1}{B(1+y/m)^2} \sum_{t=1}^{n} \frac{t(t+1) \cdot C_t}{m^2 \cdot (1+y/m)^t}
\end{equation}

\subsection{数值计算结果}

通过数值计算得到:
\begin{align}
B &= \$100.00 \\
D_{Mac} &= 13.39 \text{ 年} \\
D_{Mod} &= \frac{13.39}{1.0225} = 13.10 \\
C &= 225.09
\end{align}

\subsection{价格变化计算}

使用泰勒展开的二阶近似:
\begin{equation}
\frac{\Delta B}{B} = -D_{Mod} \Delta y + \frac{1}{2}C(\Delta y)^2
\end{equation}

代入数值:
\begin{align}
\frac{\Delta B}{B} &= -13.10 \times (-0.0001) + \frac{1}{2} \times 225.09 \times (-0.0001)^2 \\
&= 0.001310 + \frac{1}{2} \times 225.09 \times 0.00000001 \\
&= 0.001310 + 0.000001126 \\
&= 0.001311126
\end{align}

\subsection{最终结果}

\textbf{百分比价格变化:}
\begin{equation}
\frac{\Delta B}{B} = 0.1311\%
\end{equation}

\textbf{美元价格变化:}
\begin{equation}
\Delta B = 0.001311126 \times 100 = \$0.1311
\end{equation}

\textbf{效应分解:}
\begin{itemize}
    \item 久期效应:$0.1310\%$
    \item 凸性效应:$0.000113\%$
    \item 总效应:$0.1311\%$
\end{itemize}

\subsection{结论}

当20年期4.5\%国债的收益率下降1个基点时,债券价格上升约$\$0.1311$(按$\$100$面值计算),其中久期效应占主导地位,凸性效应在小幅收益率变化时可忽略不计。