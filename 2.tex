\chapter{Interest Rate}

\section{Rates}

\textbf{Fed Fund Rate :}
unsecured interbank overnight rate of interest allows banks to adjust cash on deposit with fed at the end of the day

\textbf{Repo Rate:}
FI owns security agree to sell them for X and buy them back in the future for a slightly higher price Y. Repo rate is Y -X

\textbf{Treasury Rate :}
Rate on instrument issused by a governmnet in its own currency

\textbf{LIBOR :}
rate of interest at which AA bank can borrow money on an unsecured basis from another bank

\textbf{SOFR}(Secured Overnight Financing Rate)是美国新的基准利率,用于替代LIBOR。

\begin{equation}
\text{SOFR} = \text{美国国债回购市场隔夜借贷成本的交易量加权中位数}
\end{equation}

\textbf{核心特征:}
\begin{itemize}
    \item \textbf{有担保}:以美国国债为抵押
    \item \textbf{基于实际交易}:非银行报价
    \item \textbf{交易量巨大}:约1万亿美元/天
    \item \textbf{接近无风险利率}:无信用风险
\end{itemize}


\textbf{SOFR的期限结构}

由于SOFR是隔夜利率,需要构建期限结构:

\begin{enumerate}
    \item \textbf{简单平均SOFR}:
    \begin{equation}
    \text{Simple SOFR} = \frac{1}{n}\sum_{i=1}^{n} \text{SOFR}_i
    \end{equation}
    
    \item \textbf{复利SOFR}:
    \begin{equation}
    \text{Compounded SOFR} = \left[\prod_{i=1}^{n}\left(1 + \frac{\text{SOFR}_i \times d_i}{360}\right)\right] - 1
    \end{equation}
    
    \item \textbf{期限SOFR}(Term SOFR):基于SOFR衍生品市场的前瞻性利率
\end{enumerate}


\textbf{1. 浮动利率贷款}
\begin{equation}
\text{贷款利率} = \text{Term SOFR} + \text{信用利差}
\end{equation}

\textbf{2. 利率衍生品}
\begin{itemize}
    \item SOFR期货
    \item SOFR利率互换
    \item SOFR期权
\end{itemize}

\textbf{3. 浮息债券}
\begin{equation}
\text{票息} = \text{Compounded SOFR} + \text{利差}
\end{equation}

\textbf{关键时间点}:美元LIBOR已于2023年6月30日完全停止发布,SOFR成为美国主要基准利率。

\begin{table}[h]
\centering
\caption{美国主要基准利率对比}
\begin{tabular}{|l|c|c|c|}
\hline
\textbf{特征} & \textbf{Fed Funds Rate} & \textbf{SOFR} & \textbf{Treasury Rate} \\
\hline
\hline
\textbf{全称} & Federal Funds Rate & Secured Overnight & Treasury Yield \\
 & (联邦基金利率) & Financing Rate & (国债收益率) \\
\hline
\textbf{定义} & 银行间隔夜拆借利率 & 国债回购隔夜利率 & 美国国债收益率 \\
\hline
\textbf{抵押物} & 无抵押 & 美国国债 & 不适用(直接购买) \\
\hline
\textbf{期限} & 隔夜 & 隔夜 & 多种(1个月-30年) \\
\hline
\textbf{市场参与者} & 商业银行 & 银行、对冲基金、 & 全球投资者 \\
 &  & 政府机构等 &  \\
\hline
\textbf{交易量} & 约500亿美元/天 & 约1万亿美元/天 & 约6000亿美元/天 \\
\hline
\textbf{利率类型} & 政策利率/市场利率 & 市场利率 & 市场利率 \\
\hline
\textbf{设定方式} & FOMC设定目标区间 & 市场交易决定 & 拍卖/二级市场 \\
 & 市场交易形成 & (加权中位数) & 交易决定 \\
\hline
\textbf{风险特征} & 银行信用风险 & 接近无风险 & 无风险 \\
\hline
\textbf{用途} & 货币政策工具 & 贷款/衍生品基准 & 无风险收益基准 \\
 & 短期融资基准 & LIBOR替代品 & 资产定价基础 \\
\hline
\textbf{波动性} & 低(政策引导) & 中等 & 高(期限越长越高) \\
\hline
\textbf{发布频率} & 每日 & 每日 & 实时 \\
\hline
\textbf{发布机构} & 纽约联储 & 纽约联储 & 美国财政部 \\
\hline
\hline
\textbf{典型水平} & 5.25\%-5.50\% & 5.30\% & 1个月:5.40\% \\
\textbf{(2024年)} &  &  & 10年:4.20\% \\
\hline
\textbf{相互关系} & \multicolumn{3}{c|}{Fed Funds Rate $\leq$ SOFR $\approx$ 短期Treasury Rate} \\
\hline
\end{tabular}
\end{table}

\begin{table}[H]
\centering
\caption{三种利率的应用场景对比}
\begin{tabular}{|l|p{4cm}|p{4cm}|p{4cm}|}
\hline
\textbf{应用领域} & \textbf{Fed Funds Rate} & \textbf{SOFR} & \textbf{Treasury Rate} \\
\hline
\textbf{货币政策} & 主要政策工具 & 政策传导指标 & 政策效果反映 \\
\hline
\textbf{银行业务} & 准备金管理 & 回购融资成本 & 资产配置基准 \\
 & 同业拆借定价 & 贷款定价基准 & 债券投资收益 \\
\hline
\textbf{衍生品} & Fed Funds期货 & SOFR期货/互换 & 国债期货 \\
 &  & 利率期权 & 利率互换基准 \\
\hline
\textbf{企业融资} & 间接影响 & 浮动利率贷款 & 债券定价基准 \\
 & (通过银行) & 直接挂钩 & 信用利差计算 \\
\hline
\textbf{风险管理} & 流动性风险指标 & 利率风险对冲 & 无风险利率 \\
 &  & 基准利率风险 & 久期管理 \\
\hline
\end{tabular}
\end{table}


\section{联邦基金利率的货币政策传导机制}

\subsection{政策工具体系}

美联储通过联邦基金利率(Fed Funds Rate)实施货币政策的核心机制:

\begin{figure}[H]
\centering
\begin{tikzpicture}[node distance=2cm]
\node[rectangle, draw, text width=3cm, align=center] (fomc) {FOMC决策};
\node[rectangle, draw, text width=3cm, align=center, below of=fomc] (target) {目标利率区间};
\node[rectangle, draw, text width=3cm, align=center, below of=target] (tools) {政策工具};
\node[rectangle, draw, text width=3cm, align=center, below of=tools] (market) {市场利率};

\draw[->] (fomc) -- (target);
\draw[->] (target) -- (tools);
\draw[->] (tools) -- (market);
\end{tikzpicture}
\end{figure}

\subsection{利率走廊机制}

美联储通过"利率走廊"系统引导市场利率:

\begin{equation}
\text{IORB} \geq \text{Fed Funds Rate} \geq \text{ON RRP}
\end{equation}

其中:
\begin{itemize}
    \item \textbf{IORB}(Interest on Reserve Balances):准备金利率,构成利率上限
    \item \textbf{ON RRP}(Overnight Reverse Repo):隔夜逆回购利率,构成利率下限
    \item \textbf{Fed Funds Rate}:在走廊内波动
\end{itemize}

\begin{table}[h]
\centering
\caption{利率走廊工具}
\begin{tabular}{|l|c|c|l|}
\hline
\textbf{工具} & \textbf{当前水平} & \textbf{作用} & \textbf{参与者} \\
\hline
IORB & 5.40\% & 利率上限 & 银行 \\
ON RRP & 5.30\% & 利率下限 & 货币市场基金等 \\
Fed Funds目标 & 5.25\%-5.50\% & 政策信号 & 银行间市场 \\
\hline
\end{tabular}
\end{table}

\subsection{传导路径}

\subsubsection{直接传导}

\begin{enumerate}
    \item \textbf{银行间市场}
    \begin{equation}
    \text{Fed Funds Rate} \uparrow \Rightarrow \text{银行资金成本} \uparrow
    \end{equation}
    
    \item \textbf{货币市场}
    \begin{equation}
    \text{Fed Funds Rate} \rightarrow \text{SOFR} \rightarrow \text{CP/CD利率}
    \end{equation}
    
    \item \textbf{存贷款利率}
    \begin{equation}
    \text{Prime Rate} = \text{Fed Funds Rate} + 3\%
    \end{equation}
\end{enumerate}

\subsubsection{间接传导}

\begin{table}[H]
\centering
\caption{货币政策传导渠道}
\begin{tabular}{|l|p{6cm}|p{5cm}|}
\hline
\textbf{渠道} & \textbf{传导机制} & \textbf{经济影响} \\
\hline
\textbf{利率渠道} & Fed Funds $\uparrow$ $\rightarrow$ 市场利率 $\uparrow$ & 投资下降、消费减少 \\
\hline
\textbf{信贷渠道} & 银行成本 $\uparrow$ $\rightarrow$ 贷款供给 $\downarrow$ & 信贷紧缩、经济放缓 \\
\hline
\textbf{汇率渠道} & 利率 $\uparrow$ $\rightarrow$ 美元升值 & 出口下降、进口增加 \\
\hline
\textbf{资产价格} & 利率 $\uparrow$ $\rightarrow$ 股票/房价 $\downarrow$ & 财富效应减弱 \\
\hline
\textbf{预期渠道} & 政策信号 $\rightarrow$ 通胀预期 $\downarrow$ & 消费和投资决策改变 \\
\hline
\end{tabular}
\end{table}

\subsection{政策实施工具}

\subsubsection{公开市场操作}

\begin{itemize}
    \item \textbf{回购操作(Repo)}:注入流动性,压低利率
    \item \textbf{逆回购操作(Reverse Repo)}:回收流动性,推高利率
\end{itemize}

\begin{equation}
\text{流动性过剩} \Rightarrow \text{Fed Funds Rate} < \text{目标下限} \Rightarrow \text{逆回购操作}
\end{equation}

\subsubsection{常备便利工具}

\begin{table}[H]
\centering
\begin{tabular}{|l|c|l|}
\hline
\textbf{工具} & \textbf{利率} & \textbf{功能} \\
\hline
贴现窗口(Primary Credit) & Fed Funds + 0.1\% & 紧急流动性支持 \\
常备回购便利(SRF) & Fed Funds + 0.25\% & 市场流动性保障 \\
\hline
\end{tabular}
\end{table}

\subsection{政策效果评估}

\subsubsection{泰勒规则}

美联储的利率决策可用泰勒规则近似:

\begin{equation}
r_t = r^* + \pi_t + 0.5(\pi_t - \pi^*) + 0.5(y_t - y^*)
\end{equation}

其中:
\begin{itemize}
    \item $r_t$:联邦基金利率
    \item $r^*$:中性实际利率(约0.5\%)
    \item $\pi_t$:当前通胀率
    \item $\pi^*$:目标通胀率(2\%)
    \item $y_t - y^*$:产出缺口
\end{itemize}

\subsubsection{政策时滞}

\begin{table}[h]
\centering
\caption{货币政策时滞}
\begin{tabular}{|l|c|l|}
\hline
\textbf{类型} & \textbf{时长} & \textbf{说明} \\
\hline
认知时滞 & 1-3个月 & 识别经济问题 \\
决策时滞 & 1-2个月 & FOMC会议周期 \\
实施时滞 & 即时 & 利率调整立即生效 \\
影响时滞 & 6-18个月 & 对实体经济的影响 \\
\hline
\end{tabular}
\end{table}

\subsection{案例分析:2022-2024加息周期}

\begin{table}[h]
\centering
\caption{美联储加息路径与经济影响}
\begin{tabular}{|l|c|c|c|}
\hline
\textbf{时期} & \textbf{Fed Funds} & \textbf{CPI} & \textbf{政策意图} \\
\hline
2022年3月 & 0-0.25\% & 8.5\% & 开始紧缩 \\
2022年12月 & 4.25-4.50\% & 6.5\% & 快速加息 \\
2023年7月 & 5.25-5.50\% & 3.2\% & 接近终点 \\
2024年(预期) & 5.25-5.50\% & 2.5\% & 维持限制性 \\
\hline
\end{tabular}
\end{table}

\textbf{政策效果}:
\begin{itemize}
    \item 通胀从9.1\%峰值回落至3\%附近
    \item 失业率仅小幅上升至4\%
    \item 实现了"软着陆"目标
\end{itemize}

\subsection{前瞻性指引}

美联储通过以下方式增强政策效果:

\begin{enumerate}
    \item \textbf{点阵图}:FOMC成员利率预期
    \item \textbf{经济预测}:GDP、通胀、失业率预测
    \item \textbf{政策声明}:会后声明措辞变化
    \item \textbf{主席讲话}:解释政策意图
\end{enumerate}

\begin{equation}
\text{市场预期} = f(\text{点阵图}, \text{经济数据}, \text{Fed沟通})
\end{equation}

\subsection{总结}

联邦基金利率作为货币政策核心工具:
\begin{itemize}
    \item \textbf{锚定作用}:引导整个利率体系
    \item \textbf{信号功能}:传递政策立场
    \item \textbf{传导机制}:通过多渠道影响经济
    \item \textbf{灵活调整}:应对经济周期变化
\end{itemize}


\section{利率转换公式详解}

\subsection{基本定义}

在金融计算中,同一个利率可以用不同的复利方式表示。主要包括:

\begin{itemize}
    \item $R_c$:连续复利利率(Continuously Compounded Rate)
    \item $R_m$:年复利$m$次的利率(Rate with $m$ Compounding Periods per Year)
\end{itemize}

\subsection{核心原理}

两种利率表示方式必须产生相同的终值:

\begin{equation}
\text{终值} = P \cdot e^{R_c \cdot t} = P \cdot \left(1 + \frac{R_m}{m}\right)^{m \cdot t}
\end{equation}

其中:
\begin{itemize}
    \item $P$:本金
    \item $t$:时间(年)
    \item $m$:每年复利次数
\end{itemize}

\subsection{转换公式推导}

\subsubsection{从离散复利到连续复利}

由等值条件:
\begin{equation}
e^{R_c} = \left(1 + \frac{R_m}{m}\right)^m
\end{equation}

两边取自然对数:
\begin{equation}
\boxed{R_c = m \ln\left(1 + \frac{R_m}{m}\right)}
\end{equation}

\subsubsection{从连续复利到离散复利}

由等值条件:
\begin{equation}
\left(1 + \frac{R_m}{m}\right)^m = e^{R_c}
\end{equation}

解出$R_m$:
\begin{equation}
\boxed{R_m = m\left(e^{R_c/m} - 1\right)}
\end{equation}

\subsection{常见复利频率}

\begin{table}[h]
\centering
\caption{常见复利频率及其对应的$m$值}
\begin{tabular}{|l|c|l|}
\hline
\textbf{复利频率} & \textbf{$m$值} & \textbf{说明} \\
\hline
年复利(Annual) & 1 & 每年复利一次 \\
半年复利(Semi-annual) & 2 & 每半年复利一次 \\
季度复利(Quarterly) & 4 & 每季度复利一次 \\
月复利(Monthly) & 12 & 每月复利一次 \\
日复利(Daily) & 365 & 每日复利一次 \\
连续复利(Continuous) & $\infty$ & 连续复利 \\
\hline
\end{tabular}
\end{table}

\subsection{数值示例}

\textbf{例1}:将年利率12\%(月复利)转换为连续复利利率

给定:$R_{12} = 0.12$,$m = 12$

\begin{align}
R_c &= 12 \ln\left(1 + \frac{0.12}{12}\right) = 0.1194 \text{ 或 } 11.94
\end{align}

\textbf{例2}:将连续复利10\%转换为季度复利

给定:$R_c = 0.10$,$m = 4$

\begin{align}
R_4 &= 4\left(e^{0.10/4} - 1\right) = 0.1013 \text{ 或 } 10.13\
\end{align}

\subsection{转换表}

\begin{table}[H]
\centering
\caption{10\%名义利率的不同复利方式等值转换}
\begin{tabular}{|l|c|c|}
\hline
\textbf{复利方式} & \textbf{名义利率} & \textbf{有效年利率} \\
\hline
年复利($m=1$) & 10.00\% & 10.00\% \\
半年复利($m=2$) & 10.00\% & 10.25\% \\
季度复利($m=4$) & 10.00\% & 10.38\% \\
月复利($m=12$) & 10.00\% & 10.47\% \\
日复利($m=365$) & 10.00\% & 10.52\% \\
连续复利($m=\infty$) & 10.00\% & 10.52\% \\
\hline
\end{tabular}
\end{table}

\subsection{特殊性质}

\textbf{极限关系}

当$m \to \infty$时:
\begin{equation}
\lim_{m \to \infty} \left(1 + \frac{R_m}{m}\right)^m = e^{R_c}
\end{equation}

这就是连续复利的数学定义。

\textbf{泰勒展开}

对于小的$\frac{R_m}{m}$:
\begin{equation}
R_c \approx R_m - \frac{R_m^2}{2m} + \frac{R_m^3}{3m^2} - \cdots
\end{equation}

当$R_m$较小时,$R_c \approx R_m$。


\textbf{ 债券定价:}

许多债券模型使用连续复利便于数学处理:
\begin{equation}
P = \sum_{i=1}^{n} C_i e^{-R_c t_i} + F e^{-R_c T}
\end{equation}

\section{Yiled}

\textbf{Bond Yield:}
The bond Yield is discount rate that makes the pv  = market price of bond

\textbf{Par Yield:}
the par yield for a certain maturity is the coupon rate that causes the bond price  =  face value


平价收益率(Par Yield)是使债券价格等于面值(平价发行)的票息率。对于平价债券:
\begin{itemize}
    \item 债券价格 = 面值 = 100
    \item 票息率 = 到期收益率
\end{itemize}

\subsection{平价收益率公式推导}

\subsubsection{关键变量定义}

\begin{itemize}
    \item $c$:年票息率(平价收益率)
    \item $m$:每年付息次数
    \item $n$:总付息次数
    \item $d$:\$1面值在到期日的现值(贴现因子)
    \item $A$:每个付息日\$1年金的现值
\end{itemize}

\subsubsection{债券定价公式}

平价债券的定价条件:
\begin{equation}
100 = \frac{c}{m} \times 100 \times A + 100 \times d
\end{equation}

其中:
\begin{itemize}
    \item $\frac{c}{m} \times 100$:每期票息支付
    \item $A$:年金现值因子
    \item $100 \times d$:本金的现值
\end{itemize}

\subsubsection{求解平价收益率}

整理上述等式:
\begin{align}
100 &= \frac{100c}{m} \times A + 100d \\
1 &= \frac{c}{m} \times A + d \\
1 - d &= \frac{c}{m} \times A \\
c &= \frac{(1-d)m}{A}
\end{align}

因此,平价收益率公式为:
\begin{equation}
\boxed{c = \frac{(100-100d)m}{A}}
\end{equation}

\subsection{现值因子的计算}

\subsubsection{贴现因子 $d$}

\begin{equation}
d = \frac{1}{(1+y/m)^{mn}}
\end{equation}

其中$y$是相应期限的零息票收益率。

\subsubsection{年金现值因子 $A$}

\begin{equation}
A = \sum_{i=1}^{n} \frac{1}{(1+y_i/m)^{i}}
\end{equation}

对于平坦的收益率曲线:
\begin{equation}
A = \frac{1-(1+y/m)^{-n}}{y/m}
\end{equation}

\subsection{数值示例}

根据题目给出的数据:
\begin{itemize}
    \item $m = 2$(半年付息)
    \item $d = 0.87284$
    \item $A = 3.70027$
\end{itemize}

计算平价收益率:
\begin{align}
c &= \frac{(100-100 \times 0.87284) \times 2}{3.70027} \\
&= \frac{(100-87.284) \times 2}{3.70027} \\
&= \frac{12.716 \times 2}{3.70027} \\
&= \frac{25.432}{3.70027} \\
&= 6.873\%
\end{align}

\subsection{扩展分析}

\subsubsection{与零息票收益率的关系}

平价收益率是零息票收益率的加权平均:
\begin{equation}
c \approx \frac{\sum_{i=1}^{n} w_i \cdot y_i}{\sum_{i=1}^{n} w_i}
\end{equation}

其中$w_i$是各期现金流的现值权重。

\subsubsection{收益率曲线形状的影响}

\begin{table}[H]
\centering
\caption{收益率曲线形状对平价收益率的影响}
\begin{tabular}{|l|c|l|}
\hline
\textbf{曲线形状} & \textbf{平价收益率} & \textbf{说明} \\
\hline
向上倾斜 & $<$ 长期零息票利率 & 早期现金流权重大 \\
平坦 & $=$ 零息票利率 & 所有期限利率相同 \\
向下倾斜 & $>$ 长期零息票利率 & 早期高利率影响大 \\
\hline
\end{tabular}
\end{table}

\section{常见的收益率(Yield)类型详解}

\subsection{收益率分类概览}

\begin{table}[H]
\centering
\caption{常见收益率类型一览}
\begin{tabular}{|l|l|l|}
\hline
\textbf{类别} & \textbf{收益率类型} & \textbf{主要用途} \\
\hline
\multirow{4}{*}{基础收益率} & 名义收益率 & 票息与面值比率 \\
& 当前收益率 & 票息与市价比率 \\
& 到期收益率(YTM) & 内部收益率 \\
& 赎回收益率(YTC) & 考虑赎回条款 \\
\hline
\multirow{3}{*}{市场收益率} & 零息票收益率 & 纯贴现收益率 \\
& 平价收益率 & 平价发行票息率 \\
& 远期收益率 & 未来期间收益率 \\
\hline
\multirow{3}{*}{风险调整收益率} & 信用利差 & 信用风险补偿 \\
& 期权调整利差(OAS) & 剔除期权价值 \\
& Z-利差 & 相对零息票曲线 \\
\hline
\multirow{2}{*}{其他收益率} & 实际收益率 & 剔除通胀影响 \\
& 税后收益率 & 考虑税收影响 \\
\hline
\end{tabular}
\end{table}

\subsection{基础收益率详解}

\textbf{1. 名义收益率(Nominal Yield / Coupon Yield)}

\begin{equation}
\text{名义收益率} = \frac{\text{年票息}}{\text{面值}} \times 100\%
\end{equation}

\textbf{特点}:
\begin{itemize}
    \item 固定不变,等于票息率
    \item 不考虑债券市场价格
    \item 仅反映票息支付水平
\end{itemize}

\textbf{2. 当前收益率(Current Yield)}

\begin{equation}
\text{当前收益率} = \frac{\text{年票息}}{\text{市场价格}} \times 100\%
\end{equation}

\textbf{示例}:面值\$1000,票息6\%,市价\$950
\begin{equation}
\text{当前收益率} = \frac{60}{950} = 6.32\%
\end{equation}

\textbf{3. 到期收益率(Yield to Maturity, YTM)}

债券价格等式:
\begin{equation}
P = \sum_{t=1}^{n} \frac{C}{(1+y)^t} + \frac{F}{(1+y)^n}
\end{equation}

其中$y$即为YTM,需通过数值方法求解。

\textbf{特点}:
\begin{itemize}
    \item 最常用的收益率指标
    \item 考虑所有现金流
    \item 假设票息可按YTM再投资
\end{itemize}

\textbf{4. 赎回收益率(Yield to Call, YTC)}

\begin{equation}
P = \sum_{t=1}^{n_c} \frac{C}{(1+y_c)^t} + \frac{\text{Call Price}}{(1+y_c)^{n_c}}
\end{equation}

其中$n_c$是到赎回日的期数。

\subsection{市场收益率曲线}

\textbf{1. 零息票收益率(Zero-Coupon Yield / Spot Rate)}

\begin{equation}
P = \frac{F}{(1+z_n)^n}
\end{equation}

因此:
\begin{equation}
z_n = \left(\frac{F}{P}\right)^{1/n} - 1
\end{equation}

\textbf{应用}:
\begin{itemize}
    \item 贴现因子计算
    \item 衍生品定价基础
    \item 收益率曲线构建
\end{itemize}

\textbf{2. 平价收益率(Par Yield)}

使债券平价发行的票息率:
\begin{equation}
c = \frac{1-d_n}{\sum_{i=1}^{n} d_i}
\end{equation}

其中$d_i$是第$i$期的贴现因子。

\textbf{3. 远期收益率(Forward Rate)}

从$t_1$到$t_2$的远期利率:
\begin{equation}
f_{t_1,t_2} = \left[\frac{(1+z_{t_2})^{t_2}}{(1+z_{t_1})^{t_1}}\right]^{\frac{1}{t_2-t_1}} - 1
\end{equation}

\subsection{利差类收益率}

\textbf{1. 信用利差(Credit Spread)}

\begin{equation}
\text{信用利差} = \text{公司债YTM} - \text{国债YTM}
\end{equation}

\begin{table}[H]
\centering
\caption{典型信用利差水平}
\begin{tabular}{|l|c|}
\hline
\textbf{信用评级} & \textbf{利差范围(bps)} \\
\hline
AAA & 20-50 \\
AA & 40-80 \\
A & 60-120 \\
BBB & 100-250 \\
BB (高收益) & 300-500 \\
B & 500-800 \\
\hline
\end{tabular}
\end{table}

\textbf{2. Z-利差(Zero-Volatility Spread)}

使债券价格等于市价的平行移动利差:
\begin{equation}
P = \sum_{t=1}^{n} \frac{CF_t}{(1+z_t+\text{Z-spread})^t}
\end{equation}

\textbf{3. 期权调整利差(Option-Adjusted Spread, OAS)}

\begin{equation}
\text{OAS} = \text{Z-spread} - \text{期权成本}
\end{equation}

用于含权债券的真实信用利差评估。

\subsection{特殊收益率指标}

\textbf{1. 实际收益率(Real Yield)}

费雪方程:
\begin{equation}
(1+r_{nominal}) = (1+r_{real})(1+\pi)
\end{equation}

近似公式:
\begin{equation}
r_{real} \approx r_{nominal} - \pi
\end{equation}

\textbf{2. 税后收益率(After-Tax Yield)}

\begin{equation}
\text{税后收益率} = \text{税前收益率} \times (1 - \text{税率})
\end{equation}

\textbf{3. 持有期收益率(Holding Period Return)}

\begin{equation}
HPR = \frac{P_1 - P_0 + \text{票息}}{P_0}
\end{equation}

\subsection{收益率之间的关系}

\begin{figure}[H]
\centering
\begin{tikzpicture}[node distance=2cm]
\node[rectangle, draw] (nominal) {名义收益率};
\node[rectangle, draw, below of=nominal] (current) {当前收益率};
\node[rectangle, draw, below of=current] (ytm) {到期收益率};
\node[rectangle, draw, right of=ytm, node distance=4cm] (spot) {零息票收益率};
\node[rectangle, draw, above of=spot] (par) {平价收益率};
\node[rectangle, draw, below of=spot] (forward) {远期收益率};

\draw[->] (nominal) -- (current) node[midway, right] {考虑市价};
\draw[->] (current) -- (ytm) node[midway, right] {考虑所有现金流};
\draw[<->] (ytm) -- (spot) node[midway, above] {息票效应};
\draw[<->] (spot) -- (par) node[midway, right] {加权平均};
\draw[<->] (spot) -- (forward) node[midway, right] {期限结构};
\end{tikzpicture}
\caption{各类收益率之间的关系}
\end{figure}

\subsection{实务应用指南}

\begin{table}[h]
\centering
\caption{不同场景下的收益率选择}
\begin{tabular}{|l|l|l|}
\hline
\textbf{应用场景} & \textbf{推荐收益率} & \textbf{原因} \\
\hline
债券比较 & YTM & 综合考虑所有现金流 \\
新债发行 & 平价收益率 & 确定合适票息率 \\
衍生品定价 & 零息票收益率 & 精确贴现 \\
信用分析 & OAS & 剔除期权影响 \\
投资决策 & 税后实际收益率 & 考虑税收和通胀 \\
短期交易 & 持有期收益率 & 反映实际收益 \\
\hline
\end{tabular}
\end{table}

\section{Bootstrap方法详解}

\textbf{概述}

Bootstrap方法在金融中主要指从市场可观察的债券价格中逐步推导出零息票收益率曲线(Zero-Coupon Yield Curve)或贴现因子的过程。这是构建收益率曲线的核心技术之一。

\textbf{基本原理}

\textbf{核心思想}
\begin{enumerate}
    \item 从短期债券开始,逐步推导长期利率
    \item 利用已知的短期利率来剥离长期债券中的票息影响
    \item 递归地构建完整的期限结构
\end{enumerate}

\textbf{数学基础}

债券价格可表示为:
\begin{equation}
P = \sum_{t=1}^{n} \frac{CF_t}{(1+z_t)^t}
\end{equation}

其中:
\begin{itemize}
    \item $P$:债券价格
    \item $CF_t$:第$t$期现金流
    \item $z_t$:第$t$期的零息票利率(待求)
\end{itemize}

\subsection{Bootstrap步骤详解}

\textbf{步骤1:从最短期债券开始}

对于1年期零息票债券:
\begin{equation}
P_1 = \frac{100}{1+z_1} \Rightarrow z_1 = \frac{100}{P_1} - 1
\end{equation}

\textbf{步骤2:求解2年期零息票利率}

对于2年期付息债券(假设年付息):
\begin{equation}
P_2 = \frac{C}{1+z_1} + \frac{100+C}{(1+z_2)^2}
\end{equation}

已知$z_1$,可求解$z_2$:
\begin{equation}
z_2 = \left(\frac{100+C}{P_2 - \frac{C}{1+z_1}}\right)^{1/2} - 1
\end{equation}

\textbf{步骤3:递归求解更长期限}

对于$n$年期债券:
\begin{equation}
P_n = \sum_{t=1}^{n-1} \frac{C}{(1+z_t)^t} + \frac{100+C}{(1+z_n)^n}
\end{equation}

求解$z_n$:
\begin{equation}
z_n = \left(\frac{100+C}{P_n - \sum_{t=1}^{n-1} \frac{C}{(1+z_t)^t}}\right)^{1/n} - 1
\end{equation}

\subsection{具体算例}

\begin{table}[h]
\centering
\caption{市场债券数据}
\begin{tabular}{|c|c|c|c|}
\hline
\textbf{期限(年)} & \textbf{票息率} & \textbf{价格} & \textbf{付息频率} \\
\hline
0.5 & 0\% & 97.50 & - \\
1.0 & 0\% & 95.00 & - \\
1.5 & 3\% & 98.50 & 半年 \\
2.0 & 4\% & 99.00 & 半年 \\
\hline
\end{tabular}
\end{table}

\textbf{计算过程}:

\textbf{1. 6个月零息票利率}:
\begin{equation}
z_{0.5} = \left(\frac{100}{97.50}\right)^2 - 1 = 5.13\%
\end{equation}

\textbf{2. 1年零息票利率}:
\begin{equation}
z_1 = \frac{100}{95.00} - 1 = 5.26\%
\end{equation}

\textbf{3. 1.5年零息票利率}:
\begin{align}
98.50 &= \frac{1.5}{1.0256^{0.5}} + \frac{1.5}{1.0526^{1}} + \frac{101.5}{(1+z_{1.5})^{1.5}} \\
98.50 &= 1.463 + 1.425 + \frac{101.5}{(1+z_{1.5})^{1.5}} \\
z_{1.5} &= 5.35\%
\end{align}

\textbf{4. 2年零息票利率}:类似计算得$z_2 = 5.42\%$

\subsection{与其他方法的比较}

\begin{table}[H]
\centering
\caption{收益率曲线构建方法比较}
\begin{tabular}{|l|c|c|c|c|}
\hline
\textbf{方法} & \textbf{精确性} & \textbf{平滑性} & \textbf{复杂度} & \textbf{稳健性} \\
\hline
Bootstrap & 高 & 低 & 低 & 中 \\
样条插值 & 中 & 高 & 中 & 高 \\
Nelson-Siegel & 低 & 高 & 中 & 高 \\
主成分分析 & 中 & 高 & 高 & 高 \\
\hline
\end{tabular}
\end{table}


\section{Forward Rate 与瞬时远期利率的推导与解释}

\subsection{Forward Rate(远期利率)公式}

假设市场中存在连续复利(continuously compounded)下的零息利率 $R_1$ 与 $R_2$,分别对应到期时间 $T_1$ 与 $T_2$,则从 $T_1$ 到 $T_2$ 的远期利率 $f_{T_1,T_2}$ 由无套利原理给出:

\begin{equation}
f_{T_1,T_2} = \frac{R_2 T_2 - R_1 T_1}{T_2 - T_1}
\end{equation}

\textbf{推导:}

由于连续复利下未来的终值为:

\[
e^{R_2 T_2} = e^{R_1 T_1} \cdot e^{f_{T_1,T_2} (T_2 - T_1)}
\]

两边取自然对数:

\[
R_2 T_2 = R_1 T_1 + f_{T_1,T_2}(T_2 - T_1)
\]

解得:

\[
f_{T_1,T_2} = \frac{R_2 T_2 - R_1 T_1}{T_2 - T_1}
\]

该公式表明,从当前时点看,若希望在 $[T_1, T_2]$ 时间段内签订借贷合同,则市场隐含的利率水平为 $f_{T_1,T_2}$。

\vspace{0.5em}
\textbf{形象理解:}将 $R_1 T_1$ 和 $R_2 T_2$ 理解为“总高度”,该远期利率即为“在 $[T_1, T_2]$ 段的平均坡度”。

\subsection{瞬时远期利率(Instantaneous Forward Rate)}

当时间间隔趋近于零,即 $T_2 \rightarrow T_1$ 时,我们定义从某个未来时间点 $T$ 开始,极短时间段内适用的瞬时远期利率 $f(T)$:

\begin{equation}
f(T) = R(T) + T \cdot \frac{\partial R}{\partial T}
\end{equation}

其中,$R(T)$ 表示当前观察到的 $T$ 年期连续复利零息利率。

\textbf{推导:}

从一般远期利率公式出发,令 $T_1 = T$,$T_2 = T + \Delta T$,有:

\[
f(T, T+\Delta T) = \frac{R(T+\Delta T)(T + \Delta T) - R(T)T}{\Delta T}
\]

取极限:

\[
f(T) = \lim_{\Delta T \to 0} \frac{R(T+\Delta T)(T + \Delta T) - R(T)T}{\Delta T}
= \frac{d}{dT} [R(T) \cdot T]
= R(T) + T \cdot \frac{dR}{dT}
\]

\textbf{直观解释:}

瞬时远期利率表示“从时间 $T$ 时刻开始、持续一个无限小时间段的借贷利率”。它等价于 $R(T) \cdot T$ 的导数,体现了瞬时变化率的思想。

\subsection{两种远期利率的对比}

\begin{table}[h]
\centering
\begin{tabular}{|c|c|c|}
\hline
\textbf{内容} & \textbf{远期利率 $f_{T_1,T_2}$} & \textbf{瞬时远期利率 $f(T)$} \\
\hline
定义区间 & $T_1$ 到 $T_2$ 之间的平均利率 & $T$ 时刻之后瞬间的利率 \\
\hline
数学表达 & $\displaystyle \frac{R_2 T_2 - R_1 T_1}{T_2 - T_1}$ & $\displaystyle R(T) + T \cdot \frac{dR}{dT}$ \\
\hline
推导方式 & 无套利 + 等价收益公式 & 极限思想(导数) \\
\hline
应用场景 & FRA、IRS定价、期限结构拟合 & HJM/CIR等利率建模、forward curve 构建 \\
\hline
\end{tabular}
\caption{远期利率与瞬时远期利率的对比}
\end{table}


\subsection{收益率曲线斜率与各类利率关系}

收益率曲线(Yield Curve)展示了不同期限的利率结构形态。在实际分析中,主要考察三类利率的相对大小关系:

\begin{itemize}
    \item \textbf{Forward Rate}:远期利率,表示从未来 $T_1$ 到 $T_2$ 的隐含利率。
    \item \textbf{Zero Rate}:零息利率,从当前时点到某一期限的连续复利年化收益。
    \item \textbf{Par Yield}:票面收益率,使得债券价格等于面值时的等效利率。
\end{itemize}

当收益率曲线呈不同斜率形状时,这三种利率会呈现出典型的相对大小关系:

\textbf{上升型收益率曲线(Upward Sloping Yield Curve):}

此时长期利率高于短期利率,通常出现在经济扩张或市场预期加息时期。

\begin{equation}
\text{Forward Rate} > \text{Zero Rate} > \text{Par Yield}
\end{equation}

\begin{itemize}
    \item 市场预期未来利率上升,导致远期利率比当前零息利率更高;
    \item 零息利率是从现在到某期限的平均利率;
    \item 票面利率(Par Yield)以均值贴现现金流,因此最小。
\end{itemize}

\textbf{下降型收益率曲线(Downward Sloping Yield Curve):}

此时短期利率高于长期利率,常见于经济衰退预期或宽松货币政策下。

\begin{equation}
\text{Par Yield} > \text{Zero Rate} > \text{Forward Rate}
\end{equation}

\begin{itemize}
    \item 市场预期未来利率下降,forward rate 最低;
    \item Zero Rate 折中考虑了短期高利率与长期低利率;
    \item Par Yield 权重偏重短期高利率,因此最高。
\end{itemize}

\textbf{直观解释:}

可将三类利率类比为骑车上坡/下坡的情形:

\begin{itemize}
    \item \textbf{Par Yield}:整体平均骑行速度(所有现金流平均收益率);
    \item \textbf{Zero Rate}:从起点到终点的平均坡度;
    \item \textbf{Forward Rate}:你站在某点,看未来路段的瞬时坡度(隐含未来利率)。
\end{itemize}

\textbf{应用启示:}

\begin{itemize}
    \item 利率相对关系有助于判断市场对未来利率的隐含预期;
    \item 可用于构建期限套利策略,如利用 forward rate 高于 spot rate 的期限结构套利;
    \item 在利率建模(如 HJM、LMM)中,用于构建 forward curve 的形状约束。
\end{itemize}
