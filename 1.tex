\chapter{Hedging with futures}

%——————————————————————————————————%

\section{Basis与 Basis Risk}
基差风险
基差是现货与期货之间的价差。在到期前存在基差风险

Basis = Spot  - Future 

\begin{itemize}
    \item $F_1$:建立对冲时的期货价格
    \item $F_2$:购买资产时的期货价格
    \item $S_2$:购买时的资产价格
    \item $b_2$:购买时的基差
\end{itemize}

\subsection{对冲计算}

\begin{table}[h]
\centering
\begin{tabular}{|l|l|}
\hline
资产成本 & $S_2$ \\
\hline
期货收益 & $F_2 - F_1$ \\
\hline
净支付金额 & $S_2 - (F_2 - F_1) = F_1 + b_2$ \\
\hline
\end{tabular}
\end{table}

\subsection{什么是多头对冲}

多头对冲是指\textbf{买入期货合约}来对冲未来购买现货资产时的价格上涨风险。适用于计划在未来购买某种资产的情况。

\subsection{对冲机制解析}

\textbf{资产成本}:直接购买资产的成本为 $S_2$。

\textbf{期货收益}:
\begin{itemize}
    \item 期货头寸盈利 = $F_2 - F_1$
    \item 因为是多头头寸,期货价格上涨时获利
\end{itemize}

\textbf{净支付金额}:
$$\text{净支付} = S_2 - (F_2 - F_1) = F_1 + b_2$$


\subsection{空头对冲}


\textbf{空头对冲(Short Hedge)}是指卖出期货合约来对冲未来出售现货资产时的价格下跌风险。适用于已经持有资产或确定将来会持有资产,并计划在未来某个时间出售的情况。

\textbf{变量说明}
\begin{itemize}
    \item $F_1$:建立对冲时(卖出期货时)的期货价格
    \item $F_2$:出售资产时(平仓期货时)的期货价格
    \item $S_2$:实际出售资产时的现货价格
    \item $b_2 = S_2 - F_2$:出售时的基差
\end{itemize}


\textbf{对冲机制分析}

\textbf{1. 资产销售收入}

直接出售资产获得的收入为 $S_2$

\textbf{2. 期货头寸盈亏}
\begin{itemize}
    \item 期初:卖出期货(空头),价格为 $F_1$
    \item 期末:买入期货平仓,价格为 $F_2$
    \item 期货盈利 = $F_1 - F_2$(价格下跌时获利)
\end{itemize}

\textbf{3. 净收入计算}
\begin{equation}
\text{净收入} = S_2 + (F_1 - F_2) = F_1 + b_2
\end{equation}

\textbf{假设一个农民计划3个月后出售小麦:}

\textbf{时间点1(建立对冲):}
\begin{itemize}
    \item 当前小麦期货价格 $F_1 = 200$ 元/吨
    \item 操作:卖出期货合约
\end{itemize}

\textbf{时间点2(出售小麦):}
\begin{itemize}
    \item 现货价格 $S_2 = 180$ 元/吨(下跌了)
    \item 期货价格 $F_2 = 182$ 元/吨
    \item 基差 $b_2 = 180 - 182 = -2$ 元/吨
\end{itemize}

\textbf{计算结果:}
\begin{itemize}
    \item 现货收入:180 元/吨
    \item 期货盈利:$200 - 182 = 18$ 元/吨
    \item 净收入:$180 + 18 = 198 = 200 + (-2)$ 元/吨
\end{itemize}

通过对冲,农民将收入锁定在约200元/吨的水平,有效规避了价格下跌的风险。


\textbf{空头对冲与多头对冲对比}

\begin{table}[h]
\centering
\begin{tabular}{|l|l|l|}
\hline
\textbf{对比项} & \textbf{空头对冲} & \textbf{多头对冲} \\
\hline
期货操作 & 卖出期货 & 买入期货 \\
\hline
保护目标 & 防止价格下跌 & 防止价格上涨 \\
\hline
适用情况 & 未来要出售资产 & 未来要购买资产 \\
\hline
净结果 & $F_1 + b_2$ & $F_1 + b_2$ \\
\hline
\end{tabular}
\end{table}

两种对冲的净结果形式相同,但经济含义相反:空头对冲锁定销售收入,多头对冲锁定购买成本。


%——————————————————————————————————%

\section{最优对冲比率}

\textbf{概念定义}

最优对冲比率是指为了最小化对冲组合的风险,应该对冲的风险敞口比例。它告诉我们:对于每一单位的现货头寸,应该使用多少单位的期货合约进行对冲。

\textbf{公式推导}

最优对冲比率的公式为:
\begin{equation}
h^* = \rho \frac{\sigma_S}{\sigma_F}
\end{equation}

其中:
\begin{itemize}
    \item $h^*$:最优对冲比率
    \item $\sigma_S$:现货价格变化 $\Delta S$ 的标准差
    \item $\sigma_F$:期货价格变化 $\Delta F$ 的标准差
    \item $\rho$:$\Delta S$ 和 $\Delta F$ 之间的相关系数
\end{itemize}

\textbf{推导原理}

最优对冲比率通过最小化对冲组合的方差得出。设对冲组合的价值变化为:
\begin{equation}
\Delta V = \Delta S - h \Delta F
\end{equation}

对冲组合的方差为:
\begin{equation}
\text{Var}(\Delta V) = \sigma_S^2 + h^2\sigma_F^2 - 2h\rho\sigma_S\sigma_F
\end{equation}

对 $h$ 求导并令其等于零:
\begin{equation}
\frac{\partial \text{Var}(\Delta V)}{\partial h} = 2h\sigma_F^2 - 2\rho\sigma_S\sigma_F = 0
\end{equation}

解得最优对冲比率:
\begin{equation}
h^* = \rho \frac{\sigma_S}{\sigma_F}
\end{equation}

\textbf{经济含义}

\textbf{1. 完美相关情况}($\rho = 1$)
\begin{itemize}
    \item $h^* = \frac{\sigma_S}{\sigma_F}$
    \item 对冲比率仅取决于波动率比率
    \item 如果期货价格波动是现货的2倍,则只需0.5单位期货对冲1单位现货
\end{itemize}

\textbf{2. 不完全相关情况}($0 < \rho < 1$)
\begin{itemize}
    \item 需要调整对冲比率
    \item 相关性越低,最优对冲比率越小
    \item 反映了期货不能完全对冲现货风险的事实
\end{itemize}

\textbf{实际应用示例}

\textbf{示例1:股票组合对冲}

假设投资者持有某股票组合,计划使用股指期货对冲:
\begin{itemize}
    \item 股票组合日收益率标准差:$\sigma_S = 2\%$
    \item 股指期货日收益率标准差:$\sigma_F = 1.5\%$
    \item 相关系数:$\rho = 0.9$
\end{itemize}

最优对冲比率:
\begin{equation}
h^* = 0.9 \times \frac{2\%}{1.5\%} = 1.2
\end{equation}

这意味着每100万元的股票组合,需要卖出价值120万元的股指期货。

\textbf{示例2:外汇风险对冲}

某出口企业需要对冲美元收入:
\begin{itemize}
    \item 美元现汇汇率波动率:$\sigma_S = 0.5\%$
    \item 美元期货汇率波动率:$\sigma_F = 0.48\%$
    \item 相关系数:$\rho = 0.95$
\end{itemize}

最优对冲比率:
\begin{equation}
h^* = 0.95 \times \frac{0.5\%}{0.48\%} = 0.99
\end{equation}

几乎是1:1的对冲比率。

\textbf{实践考虑因素}

\textbf{1. 参数估计}
\begin{itemize}
    \item 使用历史数据估计 $\sigma_S$、$\sigma_F$ 和 $\rho$
    \item 选择合适的时间窗口(如30天、60天或90天)
    \item 考虑参数的时变性
\end{itemize}

\textbf{2. 合约规格}
\begin{itemize}
    \item 期货合约有标准化的规格
    \item 实际对冲比率需要四舍五入到整数合约
    \item 可能存在剩余风险敞口
\end{itemize}

\textbf{3. 动态调整}
\begin{itemize}
    \item 市场条件变化时,最优对冲比率也会变化
    \item 需要定期重新计算和调整
    \item 权衡调整成本和对冲效果
\end{itemize}

\textbf{对冲效果评估}

使用最优对冲比率后,对冲组合的最小方差为:
\begin{equation}
\text{Var}(\Delta V)_{min} = \sigma_S^2(1-\rho^2)
\end{equation}

对冲有效性(Hedge Effectiveness)可以表示为:
\begin{equation}
HE = \rho^2
\end{equation}

$\rho^2$ 也称为R-squared,表示期货价格变化能够解释现货价格变化的比例。例如,如果 $\rho = 0.9$,则 $HE = 0.81$,表示81\%的现货价格风险可以通过期货对冲消除。




%——————————————————————————————————%

\section{Beta方法与Rho方法的数学关系与转换}

\subsection{基本定义与核心公式}

\textbf{Beta系数定义:}
\begin{equation}
\beta = \frac{\text{Cov}(R_p, R_m)}{\text{Var}(R_m)} = \rho_{p,m} \cdot \frac{\sigma_p}{\sigma_m}
\end{equation}

\textbf{最优对冲比率(Rho方法):}
\begin{equation}
h^* = \rho_{S,F} \cdot \frac{\sigma_S}{\sigma_F}
\end{equation}

其中:
\begin{itemize}
    \item $\rho_{p,m}$:投资组合与市场指数的相关系数
    \item $\sigma_p, \sigma_m$:投资组合和市场指数的标准差
    \item $\rho_{S,F}$:现货与期货的相关系数
    \item $\sigma_S, \sigma_F$:现货和期货的标准差
\end{itemize}

\subsection{数学等价性证明}

从协方差的定义出发:
\begin{equation}
\beta = \frac{\text{Cov}(R_p, R_m)}{\text{Var}(R_m)} = \frac{\rho_{p,m} \cdot \sigma_p \cdot \sigma_m}{\sigma_m^2} = \rho_{p,m} \cdot \frac{\sigma_p}{\sigma_m}
\end{equation}

对比两个公式结构,可得:
\begin{equation}
\boxed{\beta = h^* \quad \text{当将股票组合视为"现货",股指期货视为"期货"时}}
\end{equation}

\subsection{应用场景对比}

\begin{table}[h]
\centering
\begin{tabular}{|l|c|c|}
\hline
\textbf{特征} & \textbf{Beta方法} & \textbf{Rho方法} \\
\hline
应用场景 & 股票组合vs股指 & 商品现货vs期货 \\
风险度量 & 系统性风险敞口 & 总体价格风险 \\
计算重点 & 相对市场的敏感度 & 相关性与波动率 \\
\hline
\end{tabular}
\end{table}

\subsection{特殊情况分析}

\begin{enumerate}
    \item \textbf{完美相关}($\rho = 1$):
    \begin{equation}
    \beta = h^* = \frac{\sigma_p}{\sigma_m}
    \end{equation}
    
    \item \textbf{相同波动率}($\sigma_p = \sigma_m$):
    \begin{equation}
    \beta = h^* = \rho
    \end{equation}
    
    \item \textbf{指数基金}($\rho \approx 1, \sigma_p \approx \sigma_m$):
    \begin{equation}
    \beta = h^* \approx 1
    \end{equation}
\end{enumerate}

\subsection{实践意义}

两种方法的数学等价性揭示了对冲的统一本质:\textbf{寻找最小化风险的最优比率}。

\begin{itemize}
    \item 对于股票组合对冲,两种方法理论上等价
    \item 选择依据:数据可得性和计算便利性
    \item 交叉对冲时必须使用Rho方法
\end{itemize}

\textbf{核心结论}:Beta本质上是Rho方法在股票市场的特殊应用,当满足转换条件时,两者可以相互转换。




