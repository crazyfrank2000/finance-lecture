\chapter{interest rate Futures}

\section{公式介绍 (Formula Introduction)}

美国国库券的定价采用折扣收益率方法,其核心公式为:

\begin{equation}
P = \frac{360}{n}(100 - Y)
\end{equation}

其中:
\begin{itemize}
    \item $P$ = 报价价格 (Quoted Price)
    \item $Y$ = 每100美元面值的现金价格 (Cash Price per \$100 face value)
    \item $n$ = 到期天数 (Days to maturity)
    \item $360$ = 银行年度天数惯例 (Banking year convention)
\end{itemize}

\section{公式推导 (Formula Derivation)}

国库券的定价基于折扣机制,投资者以低于面值的价格购买,到期时按面值赎回。

\subsection{基本关系}
折扣率 $d$ 定义为:
\begin{equation}
d = \frac{\text{面值} - \text{购买价格}}{\text{面值}} \times \frac{360}{n}
\end{equation}

设面值为100美元,购买价格为 $Y$,则:
\begin{equation}
d = \frac{100 - Y}{100} \times \frac{360}{n}
\end{equation}

\subsection{报价价格计算}
在美国国库券市场中,报价价格 $P$ 实际上表示的是年化折扣率(以百分比形式),因此:
\begin{equation}
P = d \times 100 = \frac{100 - Y}{100} \times \frac{360}{n} \times 100
\end{equation}

简化得到:
\begin{equation}
\boxed{P = \frac{360}{n}(100 - Y)}
\end{equation}

\section{公式应用示例 (Application Examples)}

\subsection{示例1}
假设一张91天到期的国库券,每100美元面值的现金价格为98.5美元:

\begin{align}
P &= \frac{360}{91}(100 - 98.5)\\
&= \frac{360}{91} \times 1.5\\
&= 3.956 \times 1.5\\
&\approx 5.93\%
\end{align}

\subsection{示例2}
如果报价价格为4.2\%,91天到期,求现金价格:

\begin{align}
4.2 &= \frac{360}{91}(100 - Y)\\
4.2 &= 3.956(100 - Y)\\
100 - Y &= \frac{4.2}{3.956}\\
100 - Y &= 1.062\\
Y &= 98.938
\end{align}

因此现金价格为每100美元面值98.938美元。

\section{公式特点分析 (Formula Analysis)}

\subsection{时间因子影响}
\begin{itemize}
    \item 随着到期时间 $n$ 增加,$\frac{360}{n}$ 减小
    \item 相同的价格差异 $(100-Y)$ 会产生更低的年化报价价格
    \item 体现了时间价值的影响
\end{itemize}

\subsection{价格关系}
\begin{itemize}
    \item $Y < 100$:国库券以折扣价交易
    \item $Y$ 越小,折扣越大,报价价格 $P$ 越高
    \item 报价价格与现金价格呈反向关系
\end{itemize}

\subsection{360天惯例}
\begin{itemize}
    \item 使用360天而非365天是银行业惯例
    \item 简化计算,便于标准化
    \item 与货币市场其他工具保持一致
\end{itemize}

\section{实际应用注意事项 (Practical Considerations)}

\begin{enumerate}
    \item \textbf{市场惯例}:美国国库券市场使用这种报价方式已有数十年历史
    \item \textbf{计算精度}:实际交易中通常保留更多小数位
    \item \textbf{到期计算}:$n$ 的计算需要考虑交割日和到期日之间的实际天数
    \item \textbf{最小变动单位}:通常为0.01\%(1个基点)
\end{enumerate}

\section{与其他金融工具的比较 (Comparison with Other Instruments)}

\begin{table}[h]
\centering
\begin{tabular}{@{}lcc@{}}
\toprule
金融工具 & 报价方式 & 年度天数惯例 \\
\midrule
国库券 & 折扣收益率 & 360天 \\
商业票据 & 折扣收益率 & 360天 \\
银行承兑汇票 & 折扣收益率 & 360天 \\
国库债券 & 收益率到期日 & 365天 \\
\bottomrule
\end{tabular}
\caption{不同金融工具的报价惯例}
\end{table}

\section{结论 (Conclusion)}

美国国库券定价公式 $P = \frac{360}{n}(100 - Y)$ 是货币市场的重要工具,它:

\begin{itemize}
    \item 提供了标准化的报价方法
    \item 便于不同期限国库券之间的比较
    \item 反映了时间价值和风险的关系
    \item 与整个货币市场体系保持一致
\end{itemize}

理解这个公式对于参与货币市场交易、风险管理和投资组合构建都具有重要意义。
